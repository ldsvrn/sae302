% !TEX encoding = UTF-8 Unicode

\documentclass{article}
\usepackage[british]{babel}
\author{Louis DESVERNOIS}
\title{%
    SAÉ3.02 \\
    \large Developer documentation}
% \date{9 Juin 2022}
\usepackage[left=2.5cm,right=2.5cm,top=2.5cm,bottom=2.5cm]{geometry}
\usepackage{subcaption}
\usepackage{listings}
\usepackage{minted}
\usepackage{graphicx}
\usepackage[T1]{fontenc}
\usepackage[colorlinks=true,linkcolor=black,anchorcolor=black,citecolor=black,filecolor=black,menucolor=black,runcolor=black,urlcolor=black]{hyperref}

\setcounter{tocdepth}{2} % pour la profondeur de la ToC

\usepackage{fancyhdr}
\pagestyle{fancy}
\fancyhf{}
\renewcommand{\headrulewidth}{0pt}
\rfoot{\thepage}
\lfoot{SAÉ3.02: Louis DESVERNOIS}

\setlength{\parindent}{0ex}

%\renewcommand{\listoflistingscaption}{Table des codes}
%\renewcommand{\listingscaption}{Code}

\begin{document}

\maketitle
\tableofcontents
\listoffigures
\listoflistings

\newpage
\section{Introduction}
This document is the developer documentation for the remote control program made
for the SAÉ3.04. On both the server and client, setting the \verb|DEBUG|
constant in the beginning of each file to \verb|True| will change some default behaviour 

\section{Server}
The server consists of two Python files, \verb|main.py| which contains the main
server class and \verb|action.py| which is used to obtain information about the
machine and execute commands.

\subsection{Server class}
The server is implemented using a Python class and is relatively simple, as it
only accepts one client at a time.

\begin{listing}[H]
    \begin{minted}[breaklines, linenos]{python}
class Server:
    def __init__(self, host: tuple):
        self.host = host
        self.killed = False
    \end{minted}
    \caption{Server constructor}
    \label{serv:init}
\end{listing}

The server class only takes a tuple \verb|(host, port)| as an argument. It does
not connect automatically, instead the \verb|start()| method has to be used,
this allows for a "clean" shutdown of the server if we except for a
\verb|KeyboardInterrupt|.

\begin{listing}[H]
    \begin{minted}[breaklines, linenos]{python}
if __name__ == "__main__":
    server = Server((host, port))
    try:
        server.start()
    except KeyboardInterrupt:
        logging.info("KeyboardInterrupt: killing server...")
        server.kill()
    \end{minted}
    \caption{Starting the server}
    \label{serv:start}
\end{listing}

Once the \verb|start()| method is called, the server creates the socket, listens
on the specified port. Once a client is connected, it will wait for incoming
messages.

\begin{listing}[H]
    \begin{minted}[breaklines, linenos]{python}
def start(self):
    while not self.killed:
        self.server = socket.socket()
        # While True loop
        self.__bind(self.host)
        self.server.listen(1)
        message = ""
        while not self.killed and message != "reset":
            self.client, addr = self.server.accept()
            message = ""  # reset so we can reconnect
            while (
                not self.killed and message != "reset" and message != "disconnect"
            ):
                # Here we wait for a message
                self.__handle(message, addr)
            # Close connection to client
        # Close the server
    # Kill the process
    \end{minted}
    \caption{start() method simplified}
    \label{serv:startmethod}
\end{listing}
The server will try to indefinitely bind itself to the specified port (at line 5
in Listing \ref{serv:startmethod}), this is done to ensure that the server can rebind to
the port after a reset.

Once a message is received it is sent to the \verb|__handle(message, addr)|
(line 15), this method serves no purposes other than code readability and
maintainability. \emph{This is where new features would be added.}

\begin{listing}[H]
    \begin{minted}[breaklines, linenos]{python}
def __handle(self, message: str, addr: tuple):
    if message == "kill":
        logging.info(f"Kill requested by {addr}...")
        self.killed = True  # avoid adding a condition to while loops
    elif message == "reset":
        logging.info(f"Client at {addr} requested a reset.")
    elif message == "info":
        self.client.send(("info" + json.dumps(actions.get_all())).encode())
    elif message[:7] == "command":
        command = json.loads(message[7:])
        rep = "cmmd"
        if command["shell"] == "dos":
            if sys.platform == "win32":
                rep += actions.send_command(command["com"], "dos")
            else:
                rep += "Cannot execute a DOS command on this operating system."
        # ... More elif to handle other OSs
        self.client.send(rep.encode())
    \end{minted}
    \caption{Handle method simplified}
    \label{serv:handlemethod}
\end{listing}

The first two conditions do not do much except print a log in the console.
However, if the server receive "info" from a client, it replies with a JSON
encoded dict object containing information about the machine (the info is
gathered using \verb|action.py|). 

If a message starts with the word "command", the server will try executing the
given command if the shell selected by the user is available and send the output
back to the client.
\section{Client}
The client consists of two classes the spans across two python files,
\verb|main.py| contains the GUI, and the file \verb |connection.py| is the
server connection, which is a class that allows the client to connect to
multiple servers.

\subsection{Connection class}
This class handles all the communication to the server, including sockets and
all actions. The connection object takes an IP address, a port as well as two
GUI objects to write information to. This makes showing data to the user as soon
the message is received.

\subsubsection{Initialization}
\begin{listing}[H]
    \begin{minted}[breaklines, linenos]{python}
def __init__(
    self, host: str, port: int, label_info: QLabel, label_command: QTextBrowser
) -> None:
    self.client = socket.socket()
    self.msgsrv = ""
    self.addr = (host, port)
    self.info = {}

    self.label_info = label_info
    self.label_command = label_command

    self.__connect()
    self.send("info")
    \end{minted}
    \caption{Connection init method}
    \label{client:init}
\end{listing}

\verb|__connect()| is used to connect to the server, having a separate
method allows reconnecting to the server after a disconnect. Once we are
connected to the server, an "info" request is automatically sent\footnote{This
can cause the client to hang at startup since the connection is not threaded}. 


\begin{listing}[H]
    \begin{minted}[breaklines, linenos]{python}
def __connect(self) -> None:
    self.client.connect(self.addr)
    # Flag to kill the handle thread
    self.__killed = False
    self.msgsrv = ""
    # Starting handle thread for incoming messages
    client_handler = threading.Thread(target=self.__handle, args=[self.client])
    client_handler.start()
    \end{minted}
    \caption{\_\_connect method}
    \label{client:connect}
\end{listing}

This method connects to the server and start the receive thread, the
\verb|self.killed| variable is used as a condition to keep the reception
running, allowing it to be killed easily.

\subsubsection{Threaded code}
\begin{listing}[H]
    \begin{minted}[breaklines, linenos]{python}
def __handle(self, conn) -> None:
    while self.msgsrv != "kill" and self.msgsrv != "reset" and not self.__killed:
        try:
            self.msgsrv = conn.recv(4096)
        except Exception as e:
            logging.error(f"Receive failed: {e}")
            break
        logging.debug(f"Size of the recieved message is {len(self.msgsrv)}")
        if not self.msgsrv:
            break  # prevents infinite loop on disconnect, auto disconnect clients
        self.msgsrv = self.msgsrv.decode()
        logging.info(f"Message from {self.addr}: {self.msgsrv}")

        if self.msgsrv[:4] == "info":
            self.info = json.loads(self.msgsrv[4:])
            logging.info("Got the server information.")
            self.label_info.setText(self._info_string())
        elif self.msgsrv[:4] == "cmmd":
            logging.info("Got a command output from the server.")
            self.label_command.append(self.msgsrv[4:])

    logging.debug(f"Closing handle thread for {self.addr}")
    self.client.close()
    self.__killed = True
    \end{minted}
    \caption{Threaded code}
    \label{client:thread}
\end{listing}

The client receives as long as the server is not killed, reset or the
\verb|self.killed| variable is not set to \verb|True|. Once the type of message
is detected, we log the message and set the text in one of the GUI element the
class has access to\footnote{The \_info\_string() method returns a formatted
string of the raw JSON data received from the server}. The \verb|send| method
checks if the connection is available before transmitting data.

Additional methods are available to the client such as \verb|disconnect|,
\verb|reconnect|, \verb|kill| or \verb|reset|.

\subsection{GUI}
This application is using tabs easily access servers with one window. Each tab
is its own \verb|QHBoxLayout| that contains two \verb|QGridLayout|. The widgets
and the connection are stored in a \verb|dict| that is appended to a list (of
tabs). The \verb|_create_tab()| is handling all the actual widget placement.

\subsubsection{Threaded code}
\begin{listing}[H]
    \begin{minted}[breaklines, linenos]{python}
def _create_tab(self, name: str, ip: str, port: int):
    Label_info = QLabel("Placeholder\nPlaceholder\nPlaceholder\nPlaceholder\nPlaceholder")
    TextBrowser_resultcommand = QTextBrowser()
    # Never crash when connectiong to a server, instead send notification to user
    try:
        logging.info(f"Connecting to {name}, {ip}:{port}...")
        conn = connection.Connection(
            ip, port, Label_info, TextBrowser_resultcommand
        )
    except Exception as e:
        logging.error(f"Connection to {name}, {ip}:{port} failed! ({e})")
        self.error_box(
            e, f"Connection to {name} ({ip}:{port}) failed!"
        )
    else:
        self.tabs.append(
            {
                "widget": QWidget(),
                "widget_left": QWidget(),
                "widget_right": QWidget(),
                "Button_info": QPushButton("Refresh information"),
                "Label_info": Label_info,
                "ComboBox_shell": QComboBox(),
                "LineEdit_sendcommand": QLineEdit(),
                "Button_clear": QPushButton("Clear"),
                "TextBrowser_resultcommand": TextBrowser_resultcommand,
                "Button_disconnect": QPushButton("Disconnect"),
                "Button_kill": QPushButton("Kill"),
                "Button_reset": QPushButton("Reset"),
                "Button_reco": QPushButton("Reconnect"),
            }
        )
        tab = self.tabs[-1]
        # We then use the tab variable to access all of the tab's elements
        ...
    \end{minted}
    \caption{Beginning of the \_create\_tab() method}
    \label{client:createtab}
\end{listing}

First we create a \verb|QLabel| and a \verb|QTextBrowser| to use for the
creation of the connection. For the actual connection we except all exceptions
and store the message is en \verb|e| variable, so we can notify the user of the
error without crashing the application, the tab is only created if not except
are encountered while connecting.

Other methods are briefly explained in the source code.

\end{document}